%%%%%%%%%%%%%%%%%%%% author.tex %%%%%%%%%%%%%%%%%%%%%%%%%%%%%%%%%%%
%
% sample root file for your "contribution" to a proceedings volume
%
% Use this file as a template for your own input.
%
%%%%%%%%%%%%%%%% Springer %%%%%%%%%%%%%%%%%%%%%%%%%%%%%%%%%%

\documentclass{styles/svproc}
%
% RECOMMENDED %%%%%%%%%%%%%%%%%%%%%%%%%%%%%%%%%%%%%%%%%%%%%%%%%%%
%

% to typeset URLs, URIs, and DOIs
\usepackage{url}
\def\UrlFont{\rmfamily}

\usepackage{amsmath}
\usepackage[super]{nth}
% \newcommand{\MyLLM}{REFINE-Verilog} % LORAV, MIHA-Verilog, COMBA-Coder, Mani-RTL, LORAFIV

\usepackage{myacronyms}
\usepackage{graphicx}
\usepackage{multirow}

% table
	
\usepackage{booktabs}

% % https://tex.stackexchange.com/questions/4709/how-do-i-set-a-maximum-column-width
\usepackage{array} % for defining a new column type
\usepackage{varwidth}

\def\MyLLMTitle{REFINE-Verilog}
\def\MyLLM{REFINE\nobreakdash-Verilog}

% \def\M-#1{M\nobreakdash-#1}
% \def\MyLLM{\M-{Verilog}}
\begin{document}
%
% https://tex.stackexchange.com/questions/4709/how-do-i-set-a-maximum-column-width
\newcolumntype{M}{>{\begin{varwidth}{90em}}c<{\end{varwidth}}} %M is for Maximal column

\mainmatter              % start of a contribution
%
% \title{\MyLLM: \underline{M}ulti-\underline{I}nstance \underline{H}ierarchical Fine-Tuning for Small \gls{llm} in Simple Verilog Generation Leveraging \gls{lora} \underline{A}pproach}
% \title{\MyLLM: \underline{Lo}gic \underline{Ra}nking-based Fine-tuning Scheme for \glsfmtlong{slm} in Simple \underline{V}erilog Generation Leveraging \gls{lora} Technique}
% \title{\MyLLM: \underline{Lo}gic \underline{Ra}nking-based \underline{F}ine-tuning and \underline{I}nferencing Enhancement for \glsfmtlong{slm} in Partial \underline{V}erilog Generation}
\title{\MyLLMTitle: \underline{Re}liability-based \underline{F}ine-tuning and \underline{In}ference \underline{E}nhancement for \glsfmtlong{slm} in Partial Verilog Generation}
%
\titlerunning{Fine-Tuning and Inference for \glsfmtshort{slm} in Partial Verilog Generation}  % abbreviated title (for running head)

%                                     also used for the TOC unless
%                                     \toctitle is used
%
\author{Vu-Minh-Thanh Nguyen\inst{1,2}
\and Ngoc-Thien-Kim Nguyen\inst{1,2}
\and Duc-Hung Le\inst{1,2}}
%
\authorrunning{Vu-Minh-Thanh Nguyen et al.} % abbreviated author list (for running head)
%
%%%% list of authors for the TOC (use if author list has to be modified)
\tocauthor{Vu-Minh-Thanh Nguyen, Duc-Phu Do, Ngoc-Thien-Kim Nguyen, and Duc-Hung Le}
%
\institute{Faculty Electronics and Telecommunications, University of Science, \\ Ho Chi Minh City, Vietnam \and Vietnam National University, Ho Chi Minh City, Vietnam\\
\email{ldhung@hcmus.edu.vn}}

% \glsunset{llm}
\glsunset{lora}
\maketitle              % typeset the title of the contribution

\begin{abstract}
%
% % Basis of LLM
% \gls{llm} has shown promising potential in various aspects of \gls{rtl} design. However, partial \gls{rtl} generation still faces many challenges in reliability and reveals many unexpected logic behaviors.
% %
% % How do I address this?
% To address these challenges, we introduce \MyLLM, a comprehensive construction in both fine-tuning and inference for partial Verilog generation.
% % What is my technique?
% In dataset construction, we implement ranking-based logic with loss weighting to form expected behaviors of \gls{llm} in fine-tuning with various Verilog complexities. In Verilog generation process, we implement dedicated techniques to resolve unusual results in prompting and logic behaviors of \gls{llm} output.
% % 
% % Result
% On applying \MyLLM on DeepSeek-Coder-7B, our 7B-sized model outperforms other commercial and open-source \gls{llm} in the task Code Completion of VerilogEval V2.
% % 
% % What is the contribution of this work?
% Finally, \MyLLM constructs strong foundation in partial Verilog generation of further hybrid architecture for \gls{llm}-powered\gls{rtl} design.
% % 
% Our public \gls{llm} and \break source code can be accessed at \url{https://huggingface.co/nvmthanhhcmus/LORAV-DeepSeek-Coder}.
% 
% Paraphrase
\glspl{llm} have demonstrated strong potential across multiple stages of \gls{rtl} design, but partial \gls{rtl} generation still suffers from reliability issues and can exhibit unforeseen logic behaviors. To overcome these limitations, we propose \MyLLM, a comprehensive framework that constructs both fine-tuning and inference for partial Verilog generation. For dataset construction, we adopt a ranking-based strategy with loss weighting to encourage the desired \gls{llm} behavior during fine-tuning across a range of Verilog complexities. During generation, we further apply specialized methods to mitigate anomalous prompting outcomes and incorrect logic patterns in \gls{llm} outputs. When applied to DeepSeek-Coder-7B, our 7B model surpasses both commercial and open-source \glspl{llm} on the VerilogEval V2 at the Code Completion task. Overall, \MyLLM\ establishes a solid foundation for partial Verilog generation and supports future hybrid architectures, such as agentic \glspl{llm}, for \gls{llm}-powered \gls{rtl} design. Our public \gls{llm} and source code are available at \break\url{https://huggingface.co/nvmthanhhcmus/LORAV-DeepSeek-Coder}.

% 
% 
% 
% How to ensure reliability in partial Verilog generation?
% How to reduce unprecedented logic behavior in Verilog dataset in fine-tuning?

% 


%
%%%%%%%
% What are easy tasks in Verilog coding?
% \gls{llm} in easy tasks. Especially, generation of single-module Verilog designs or designs with 2-to-3 instantiations.

% Fine-tuning scheme for simple Verilog generation based on logic ranking in combination with \gls{lora} technique.

\keywords{Verilog Generation, Loss Weighting, Small Language Model, Partial RTL Design}
\end{abstract}
%
\glsresetall

\section{Introduction}

% 
% What is LLM? What is its effect on the whole world?
% 1
\gls{llm} is an unprecedented innovation in many aspects of the world. Moreover, the revolution of \gls{llm} has promoted many efforts on applying \gls{llm} abilities to achieve effectiveness, performance, and efficiency. The adoption of \gls{llm} is very diverse in various applications, such as natural language generation, speech processing, code generation, or image classification~\cite{10.1145/3639372,cui-etal-2025-recent,10.1145/3641289,yin2024survey}.
% 
% What are the effects of LLM on agiel hardware designs?
In fields of engineering, the \gls{llm} usage is very diverse in various procedures ranging from software to hardware~\cite{10.1145/3695988,10876858}. In particular, in hardware engineering, researchers start to explore \gls{llm} capability in hardware design, especially in \gls{rtl} design and verification~\cite{chang2023improving}. Based on the power of \gls{llm}, many potentials are revealed in the automated generation of \gls{rtl} design and testbench~\cite{joel2024survey}.

% 
% Why "Simple" in Verilog Generation?
% Why is "Simple" in Verilog Generation matter?
% Why not Complex task/design in Verilog generation?
Regarding handling complex task in \gls{rtl} design, complicated settings in promting and task decomposition are required \cite{tang2025hivegen}.

% 
%%%%%%%%%%%
% How to adopt LLM in RTL designs?
% 		○ How to use LLM effectively in agiel RTL generation?
% 		○ Is it related to prompting and fine-tuning?
% 		○ How is prompting effective in LLM-powered RTL design?
On using \gls{llm} effectively in agile \gls{rtl} design, the two major approaches are prompt engineering based on \gls{icl}~\cite{lu2024rtllm,gubbi2025prompting,bhattacharyya2024llm,docomba} and instruction fine-tuning~\cite{liu2025rtlcoder,10691738,11133406}. In the prompting approach, immediate effectiveness and performance in syntactic and functional correctness of \gls{rtl} design are easy to achieve by constructing logic definitions and descriptions in the context of generative \gls{llm}, such as GPT-4, Claude 3.5 Sonet, or DeepSeek-R1. Additionally, these generative models with enormous parameter size

% 
% Our contribution


% 
% %%%%%%%%%%%%%%%%%%%%%%%%
% What am I doing?
% Why do I do this stuff?
% Why train this LLM? Why not others?
% Is the DeepSeek Coder a good basis to start from coding's perspective?
% Is it good for instruction tuning for domain-specific tasks in Verilog generation?
% Is the way I fine-tune and customize this meaningful to techniques in constructing an LLM for Verilog generation?
% But what is the root basis to use LLM in such Verilog generation?
% Do the base models perform well in coding, especially in Verilog coding?
% Does it have big efforts in instructing LLM to make it more specific in Verilog generation?
% Are there any bases from existing research papers for this?
% Why does everyone use LLM? Why should and shouldn't they?
% Does LLM facilitate anything in humans' stuff?
% Does LLM boost or make something quickly and effectively?

\gls{llm}~\cite{10.1145/3641289}.

\gls{slm}~\cite{10.1145/3768165}

% 
%%%%%%%%%%%%%%%%%%%%%%
Why use LLM?
\gls{llm} is good at what?
Why not using \gls{llm}?
How does \gls{llm} facilitate generation stuff?
Are \glspl{sllm} able to serve complex tasks, especially in \gls{rtl} design?
Is LLM good? Is it reliable? And why is very enormous to all aspects of the world?
Why is it used many, regularly?
%%% Currently, \glspl{llm} have become an astounding explosion in almost all aspects of the world~\cite{YAO2024100211}. Due to its enormous benefits, many \glspl{llm}, such as ChatGPT, Claude, etc., have been quickly become popular in many aspects of both academia and industry~\cite{fi15060192}.


So, the logic is good to be a consideration for fine-tuning LLM
But are there any bases to perform LLM training based on logic-specific metrics?
Why training based on quality evaluation, from LLM, is not a good consideration?

Why do I have to optimize fine-tuning for low-level logic modules?
Why not train all bunch of Verilog samples in the same scheme, in the same configuration of LoRa fine-tuning?



% Aplaca Prompt Template~\cite{taori2023alpaca} for instruction fine-tuning.

What is the average number of tokens for the module in higher logic complexity?
What is the average number of tokens for the module description and code?

Pyranet dataset~\cite{11133406}.

% 
% Is it an easy process for the whole Verilog modules?
% 
In order to construct synthesis statistics for a large number of $692K$ Verilog modules of the dataset Pyranet, with specific limits in time and memory.
% 
We perform a long-run process for logic synthesis of $692K$ Verilog from the dataset. In order to prevent out-of-resource, a multi-process and resource-constrained configuration for each Verilog module. Based on this, each synthesis process of a Verilog module is limited to a 5-minute timeout and $2 GB$ of memory.
% 
Within our synthesis scheme, statistics, the highest logic number is $1.1M$, related to a module.
statistics, the total token
% 
Following Pyranet fine-tuning scheme, we apply \textit{Curriculum Learning} to our data after categorization with loss weighting for each tier of logic complexities.
Table \ref{tab:logic_statistic}
\section{Results and Discussions}

Deepseek-Coder-7B-Instruct v1.5~\cite{guo2024deepseekcoder}
%  \begin{table}[!htb]
%     \centering
%     \caption{Performance Benchmarks of \glspl{sllm} in Verilog Generation with VerilogEval~\cite{10323812,10.1145/3718088}.}
%     \begin{tabular}{|c|c|c|c|}
%         \hline
%         Models& $pass@1$& $pass@5$& $pass@10$\\\hline
%         MG-Verilog-CodeLlama-7B~\cite{10691738}& 54.5 &60 &63\\\hline
%         RTLCoder-DeepSeek~\cite{liu2025rtlcoder}& 61.2 &76.5 &81.8\\\hline
%         DeepSeek-Coder-7B-Instruct-PyraNet~\cite{11133406} &77.6 &84.6 &89.5\\\hline
%         \hline
%         Mistral-Instruct 7B-4Bit& 16.67& 17.18& 16.47\\\hline
%         CodeLLama-Instruct 7B-4Bit& 10.9& 10.77& 9.23 \\\hline
%         CodeLLama-Instruct-COMBA 7B-4Bit& 10.9& 13.21& 12.82 \\\hline
%         DeepSeek-Coder-COMBA 7B& 0.1& 0.1& 0.1 \\\hline
%         \MyLLM-DeepSeek-Coder 7B& 33.33& 33.33& 33.33 \\\hline
%     \end{tabular}
%     \label{tab:placeholder}
% \end{table}

\begin{table}[]
    \centering
    \caption{Performance Benchmarks of \glspl{sllm} in Verilog Generation with VerilogEval V2~\cite{10.1145/3718088}.}
    \label{tab:tab1}
\begin{tabular}{|c|c|c|c|c|c|}
\toprule
\bfseries Model Name & \bfseries Model Size & \multicolumn{4}{c|}{\bfseries Task: Code Completion} \\ \midrule
\multicolumn{2}{|r|}{In-context Learning Examples:} & \multicolumn{2}{c}{Zero Shot} & \multicolumn{2}{c|}{One Shot} \\
\multicolumn{2}{|r|}{Temperature:}  & T = 0 & T = 0.8 & T = 0 & T = 0.8 \\
% Model Name &  &  &  &  &  \\
\midrule
GPT-4o~\cite{gpt4o} & Undisclosed & 59.0\% & 56.1\% & 62.8\% & 60.7\% \\ \midrule
GPT-4 Turbo~\cite{gpt4_turbo_announce} & Undisclosed & 53.9\% & 49.8\% & 59.6\% & 59.5\% \\ \midrule
GPT-4~\cite{gpt4} & Undisclosed & 42.3\% & 41.6\% & 51.3\% & 50.1\% \\ \midrule
Mistral Large~\cite{ai_au_2024} & Undisclosed & 34.0\% & 33.1\% & 44.2\% & 42.7\% \\ \midrule
Llama3.1~\cite{noauthor_meta-llamallama3_2024} & 405B & 56.4\% & 57.0\% & 59.6\% & 57.9\% \\ \midrule
Llama3.1~\cite{noauthor_meta-llamallama3_2024} & 70B & 35.3\% & 36.3\% & 34.0\% & 33.0\% \\ \midrule
Llama3~\cite{noauthor_meta-llamallama3_2024} & 70B & 37.8\% & 39.1\% & 36.5\% & 36.5\% \\ \midrule
Llama2~\cite{noauthor_meta-llamallama3_2024} & 70B & 1.3\% & 1.7\% & 15.4\% & 13.3\% \\ \midrule
CodeLlama~\cite{noauthor_meta-llamacodellama-70b-instruct-hf_2024} & 70B & 37.2\% & 29.0\% & 41.7\% & 27.4\% \\ \midrule
DeepSeek Coder~\cite{guo2024deepseekcoder} & 33B & 25.0\% & 29.3\% & 42.3\% & 37.5\% \\ \midrule
Llama3.1~\cite{noauthor_meta-llamallama3_2024} & 8B & 2.6\% & 4.9\% & 10.9\% & 12.8\% \\ \midrule
CodeGemma~\cite{noauthor_googlecodegemma-7b_2024} & 7B & 8.3\% & 8.7\% & 19.9\% & 16.2\% \\ \midrule
DeepSeek Coder~\cite{guo2024deepseekcoder} & 6.7B & 24.4\% & 21.0\% & 33.3\% & 30.3\% \\ \midrule
RTL-Coder~\cite{liu2025rtlcoder} & 6.7B & 35.9\% & 31.5\% & 37.2\% & 32.6\% \\ \toprule\bottomrule

\MyLLM-DeepSeek-Coder & 7B & \bfseries 67.3\% & \bfseries 62.6\% & \bfseries 70.5\% & \bfseries 62.3\% \\\toprule\bottomrule

\MyLLM-DeepSeek-Coder & 7B & \bfseries 51.9\% & \bfseries 55.3\%& \bfseries 55.1\%& \bfseries 55.2\%\\
\midrule
\MyLLM-DeepSeek-Coder & 7B & \bfseries 53.2\% & \bfseries 53.1\%& \bfseries 59.6\%& \bfseries 52.4\%\\
\midrule
\MyLLM-DeepSeek-Coder & 7B & \bfseries 49.4\% & \bfseries 49.5\%& \bfseries 53.2\%& \bfseries 50.8\%\\
\midrule
\MyLLM-DeepSeek-Coder & 7B & \bfseries 46.8\% & \bfseries 49.7\%& \bfseries 50\%& \bfseries 47\%\\
\bottomrule
\end{tabular}

{Number of samples $n = 1$ when $T = 0$ and $n = 20$ when $T = 0.8$.}
\end{table}

\begin{figure}
    \centering
    \includegraphics[width=\linewidth]{templates/figures/all_plot_VE_stat_v1.pdf}
    % \vspace{-1.5cm}
    \caption{Variability of VerilogEval Success Rate with Number of Logic Cells in the Task ``Code Completion'' (Zero Shot with $T = 0.8$ and $n = 20$).}
    \label{fig:ve_stat}
\end{figure}

\begin{figure}
    \centering
    \includegraphics[width=\linewidth]{templates/figures/all_plot_VE_stat.pdf}
    % \vspace{-1.5cm}
    \caption{Variability of VerilogEval Success Rate with Number of Logic Cells in the Task ``Code Completion'' (Zero Shot with $T = 0.8$ and $n = 20$).}
    \label{fig:ve_stat_2}
\end{figure}

\clearpage

\section{Conclusion}

\section{\ackname}

This research is funded by University of Science, VNU-HCM under grant number DTVT 2024-05.

\bibliographystyle{styles/bibtex/spmpsci.bst}
\bibliography{mybibfile}

\end{document}
